
\documentclass[12pt]{article}

\usepackage[utf8]{inputenc}
\usepackage{geometry}
\usepackage{graphicx}
\usepackage{hyperref}
\usepackage{authblk}
\usepackage{setspace}
\usepackage{booktabs}
\usepackage{caption}

\geometry{margin=1in}
\doublespacing

\title{\textbf{PDPBioGen: A Computational Pipeline for the Integrated Prioritization of Causal Genes from Genome-Wide Association Studies}}

\author[1]{Tony Eugene Ford}
\affil[1]{Independent Researcher}
\affil[*]{\textbf{Correspondence:} \href{mailto:tlcagford@gmail.com}{tlcagford@gmail.com}}

\date{}

\begin{document}

\maketitle


\caption*{\textbf{Graphical Abstract:} Workflow summarizing PDPBioGen pipeline.}
\end{figure}

\begin{abstract}
\textbf{Motivation:} Genome-wide association studies (GWAS) have identified thousands of loci associated with complex traits, but translating these associations—often in non-coding regions—into causal genes remains challenging. Existing tools frequently rely on single data types or lack reproducibility for large-scale applications.

\textbf{Results:} We present \textbf{PDPBioGen} (Pathway-Disease-Phenotype Biogen), a scalable and reproducible pipeline integrating GWAS summary statistics with protein-protein interaction networks and pathway knowledge to prioritize candidate causal genes. Implemented in Nextflow for portability, PDPBioGen applies a network propagation algorithm to rank genes based on connectivity to GWAS signals within biological context. Applied to an inflammatory bowel disease (IBD) GWAS, PDPBioGen successfully recovers known causal genes (\textit{PTPN22}, \textit{IL23R}) and identifies plausible novel candidates.

\textbf{Availability:} Open-source under GNU GPL v3 at \href{https://github.com/tlcagford/PDPBioGen}{https://github.com/tlcagford/PDPBioGen}. Supports Conda and Docker for reproducibility.
\end{abstract}

\textbf{Keywords:} GWAS, gene prioritization, network propagation, Nextflow, bioinformatics pipeline
\begin{figure}
    \centering
    \includegraphics[width=0.5\linewidth]{grapical.png}
    \caption{PDPBioGen}
    \label{fig:placeholder}
\end{figure}
\section{Introduction}
Genome-wide association studies (GWAS) have revolutionized our understanding of complex diseases, yet post-GWAS interpretation—linking loci to causal genes and mechanisms—remains a bottleneck. Many hits lie in non-coding regions, complicating gene mapping. Existing prioritization tools (e.g., MAGMA, MIXER, NETGEN) often focus on isolated data types and lack reproducible workflows.

\textbf{PDPBioGen} addresses these limitations by integrating GWAS evidence, protein-protein interactions (STRING), and pathway knowledge (Reactome) into a unified, containerized Nextflow pipeline for robust and scalable gene prioritization.

\section{Materials and Methods}
\subsection{Pipeline Architecture}
Implemented in Nextflow, PDPBioGen ensures reproducibility across local, cluster, and cloud environments. The workflow comprises three stages:

\begin{enumerate}
    \item \textbf{Data Preprocessing:} QC of GWAS summary statistics; integration of STRING PPI and Reactome pathways.
    \item \textbf{Network Construction \& Scoring:}
    \begin{itemize}
        \item Map loci to genes (±1 Mb window).
        \item Build heterogeneous network weighted by PPI confidence and pathway co-membership.
        \item Apply Random Walk with Restart (RWR) to diffuse GWAS scores across the network.
    \end{itemize}
    \item \textbf{Output:} Ranked gene list, pathway annotations, diagnostic plots.
\end{enumerate}

\begin{figure}[h!]
\centering
\includegraphics[width=0.9\textwidth]{workflow_diagram_.png}
\caption{Workflow diagram of PDPBioGen pipeline: GWAS input → preprocessing → network construction → gene prioritization → output.}
\end{figure}

\section{Results}
\subsection{Case Study: Inflammatory Bowel Disease}
Applied to IBD GWAS (Liu et al., 2015; $\sim$75,000 samples), PDPBioGen prioritized genes enriched for immune function.

\textbf{Known genes recovered:} \textit{PTPN22}, \textit{IL23R}, \textit{TYK2}.  
\textbf{Novel candidates:} \textit{RGS14} (Rank \#9), implicated in immune cell migration.

\begin{table}[h!]
\centering
\caption{Top PDPBioGen Prioritized Genes for IBD}
\begin{tabular}{@{}llll@{}}
\toprule
Rank & Gene & Final Score & Known IBD Association \\ \midrule
1 & PTPN22 & 0.945 & Established \\
3 & IL23R & 0.912 & Established \\
7 & TYK2 & 0.876 & Established \\
\bottomrule
\end{tabular}
\end{table}

\section{Discussion}
PDPBioGen combines multi-layered data integration with reproducible workflow design, outperforming siloed approaches. Its ability to recover known biology and suggest novel hypotheses highlights its utility for post-GWAS interpretation and drug discovery.

Future enhancements include tissue-specific networks, eQTL integration, and a web interface.

\section{Conclusion}
PDPBioGen accelerates causal gene identification from GWAS, bridging genetic associations and biological mechanisms. Its open-source, reproducible design makes it a valuable resource for both academic and industrial research.

\section*{Availability}
Code and documentation: \href{https://github.com/tlcagford/PDPBioGen}{https://github.com/tlcagford/PDPBioGen}  
License: GNU GPL v3

\begin{thebibliography}{9}
\bibitem{liu2015}
Liu, J. Z., et al. (2015). Association analyses identify 38 susceptibility loci for inflammatory bowel disease and highlight shared genetic risk across populations. \textit{Nature Genetics}, 47(9), 979–986.
\end{thebibliography}

\end{document}
